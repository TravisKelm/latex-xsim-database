\documentclass{article}
\usepackage[clear-aux,no-files]{xsim}
% clear-aux reduces the probability of faulty exercises after document changes
% no-files means that exercises and solutions will NOT be written to individual external files

\usepackage{enumitem} % customized lists
\usepackage{fmtcount} % commands that display the value of a LaTeX counter in a variety of formats (ordinal, text, hexadecimal, decimal, octal, binary etc)
\usepackage{amssymb} % math stuff


\newenvironment{myBlock}
    {\begin{center}
    \begin{tabular}{|p{0.9\textwidth}|}
    \hline\\
    }
    {
    \\\\\hline
    \end{tabular}
    \end{center}
    }


\begin{document}

If you have this code:

\begin{myBlock}
\begin{verbatim}
\begin{exercise}
  This is a very basic question.
\end{exercise}
\begin{solution}
  This is a very basic solution.
\end{solution}
\end{verbatim}
\end{myBlock}


Then you will get this output:

\begin{myBlock}
\begin{exercise}
  This is the default \texttt{exercise} environment.
\end{exercise}
\begin{solution}
  This is the default \texttt{solution} environment.
\end{solution}
\end{myBlock}


Note that the solution is not printed.
To print the solution you must use the \verb'print=true' option with the \verb'solution' environment:

\begin{myBlock}
\begin{verbatim}
\begin{exercise}
  This is a second default exercise.
\end{exercise}
\begin{solution}[print=true]
  This is the default solution to the second exercise.
  We are printing the solution this time.
\end{solution}
\end{verbatim}
\end{myBlock}


\begin{myBlock}
\begin{exercise}
  This is a second default exercise.
\end{exercise}
\begin{solution}[print=true]
  This is the default solution to the second exercise.
  We are printing the solution this time.
\end{solution}
\end{myBlock}


You can see here that the numbering is consistent in the sense that the first time we called \verb'\begin{exercise}' we got problem \#1 and the second time that we call \verb'\begin{exercise}', then we get problem \#2.
The counter that goes with the \texttt{exercises} environment is the \texttt{counter} ``property'' of the exercise environment that you can reference using \verb'\theexercise':

\begin{myBlock}
\begin{verbatim}
  The current exercise counter is~\theexercise.
\end{verbatim}
\end{myBlock}

\begin{myBlock}
  The current exercise counter is~\theexercise.
\end{myBlock}

What we really want to be able to do is to define our own ``exercise-like'' environments, in particular, a \emph{multiple choice} environment.

On second thought, maybe we don't have to reinvent the wheel here.
It is possible to add new properties to an exercise environment using the \verb'\DeclareExerciseProperty' command.  Maybe we can use that... 


\end{document}


%%%%%%%%%%%%%%%%%%%%%%%%%%%%%%%%%%%%%%%%%%%%%%%%%%%%%%%%%%%%%%%%%%%%
\question
Simplify $\lim_{x\to 18}\sqrt{5}$.
\vspacer

\begin{oneparchoices}
  \CorrectChoiceEmphasis{\large\bfseries\boldmath}
      \CorrectChoice $\sqrt{5}$ %\correct
      \choice $5$
      \choice $18\sqrt{5}$
      \choice $18$
      \choice None of these
\end{oneparchoices}

%\begin{solution}
%\end{solution}
%%%%%%%%%%%%%%%%%%%%%%%%%%%%%%%%%%%%%%%%%%%%%%%%%%%%%%%%%%%%%%%%%%%%