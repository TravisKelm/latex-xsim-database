\documentclass{article}
\usepackage[clear-aux]{xsim}

%%%%%%%%%%%%%%% MATH PACKAGES %%%%%%%%%%%%%%%
\usepackage{amsmath}      % math papers (\text}
%\usepackage{amssymb}      % COMPLETE set of AMSFonts (needed for \mathbb)
%\usepackage{latexsym}     % to get the 11 symbols left out of latex2e
%\usepackage{eufrak}       % Euler Fraktur package
%\usepackage{euscript}     % Euler Script package
%%%%%%%%%%%%%%%%%%%%%%%%%%%%%%%%%%%%%%%%%%%%%%%%

\DeclareExerciseCollection[difficulty=easy]{easy}
\DeclareExerciseCollection[difficulty=medium]{medium}
\DeclareExerciseTagging{difficulty}

% The following tells xsim to record info about exercises/solutions in auxiliary files
% so I can do things like point totals, lists, and filtered selection later.
\xsimsetup{collect}

\begin{document}

% DIRECTLY PRINTING AN EXERCISE
\begin{exercise}[print]
  This is an exercise that is ``outside'' of any collection.
  The \texttt{[use,print]} options mean the exercise will be enumerated and printed no matter what other specifications are made.

  Just using the option \texttt{[use]} will result in the exercise being enumerated, but \textbf{not} printed.
  (In other words, the quadratic problem below will be enumerated as \#2 even though the quadratic problem might be the first problem that actually gets printed.)

  Just using the option \texttt{[print]} will print the problem, but will not
  enumerate it.
\end{exercise}


% THE FOLLOWING ARE THE EXERCISE DEFINITIONS
% EACH EXERCISE IS GROUPED INTO A ``COLLECTION'' BASED UPON ``difficulty''
% THE EXERCISES WILL NOT BE PRINTED UNTIL THEY ARE INVOKED

%%% EASY,1
\begin{exercise}[difficulty=easy,points=1]
  State the quadratic formula.
\end{exercise}
\begin{solution}
  \[ x=\frac{-b\pm\sqrt{b^2-4ac}}{2a} \]
\end{solution}

%%% MEDIUM,1
\begin{exercise}[difficulty=medium,points=1]
  State the distance formula.
\end{exercise}
\begin{solution}
  \[ \text{distance }=\sqrt{(x-a)^2+(y-b)^2} \]
\end{solution}


%%% EASY,1
\begin{exercise}[difficulty=easy,points=1]
  State the limit definition of the derivative.
\end{exercise}
\begin{solution}
  \[ f'(x)=\lim_{h\to 0}\frac{f(x+h)-f(x)}{h} \]
\end{solution}

%%%%%%%%%%%%%%%%%%%%%%%%%%%%%%%%%%%%%%%%%%%%%%%%%%%%%%%%%%%%


%%%%% HERE IS WHERE WE DECLARE WHAT EXERCISES WE WANT PRINTED

\section{Easy}
\printcollection{easy}

\section{Medium}
\printcollection{medium}

\printsolutions[difficulty=medium]

\end{document} 